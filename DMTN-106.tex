\documentclass[DM,authoryear,toc,lsstdraft]{lsstdoc}
% lsstdoc documentation: https://lsst-texmf.lsst.io/lsstdoc.html

% Package imports go here.

% Local commands go here.

\input meta.tex

% To add a short-form title:
\title{DM Release Process}

% Optional subtitle
% \setDocSubtitle{A subtitle}

\author{%
Gabriele Comoretto
}

\setDocRef{DMTN-106}

\date{\today}

% Optional: name of the document's curator
% \setDocCurator{The Curator of this Document}

\setDocAbstract{%
Release procedure applicable to all Data Management SW products.
}

% Change history defined here.
% Order: oldest first.
% Fields: VERSION, DATE, DESCRIPTION, OWNER NAME.
% See LPM-51 for version number policy.
\setDocChangeRecord{%
  \addtohist{}{2019-02-04}{DM Release Process}{Gabriele Comoretto}
}


\begin{document}

\setDocUpstreamLocation{\url{https://github.com/lsst-dm/dmtn-106}}
\setDocUpstreamVersion{\vcsrevision}

\maketitle

\section{Introduction} \label{sec:intro}

The scope of this document is to provide a general release procedure, valid for all Software Products in the Data Management LSST subsystem. The procedure as presented here can be tailored if needed.

\subsection{Applicable Documents}

When applicable documents change a change may be required in this document.
\begin{tabbing}
AUTH-NUM\= \kill
\citeds{LDM-148} \>     DM Architecture\\
\citeds{LDM-294} \>     DM Project Management Plan   \\
%\citeds{LDM-564} \>     Data Management Releases for Verification/Integration \\
% perhaps \citell{LL:AUTH-code}\>       Software Requirements Specification for \CU,\\
\end{tabbing}

\newpage
\section{Definitions} \label{sec:definitions}

\subsection{Software Product} \label{sect:swprod}
A release is made of a SW product
A SW Product should correspond to a single repository (git package)
In the case of lsst a SW corresponds to multiple git packages but the single repository can be mimicked using a metapackage 


\siubsection{Software Release} \label{sect:swrel}
Identified by  a TAG in the SW repository
 documented with a software release note.
The TAG is created on a release branch after manual checks
A released software depends only on other released software packages.


\subsection{Software Binary Package} \label{sect:swbpkg}
Is a binary package created by building the SW with the identified release Tag 
Binary packages can be created to support multiple platforms (such as linux, osx, windows) if required.
A SW package can be created from a release candidate also, or from any other identified status  in the SW repository


\subsection{Distribution} \label{sect:distribution}
Is a collection of binaries to be deployed in a specific configuration, for operational purposes.
The same distribution, shall be used for validation and integration purposes.

\newpage
\section{Change control} \label{sect:cangecontrol}
The CCB decides the timing of a release 
Major releases shall be planned with sufficient advance notice (i.e. tied to LDM-503 Milestones)
Patch release need to be agreed by the CCB on case by case basis
CCB shall also approve the issues to be included in or back ported to a patch release.

\newpage
\section{Release note} \label{sect:relnote}
List of jira issues automatically generated from the github repo [Fix in Versions(s) field in Jira can be filled by the developer if he has the information, or can be populated a posteriori when the release is done.] 
(a Release Issue can be used also, but the native solution in Jira for this use case is the Fix in Version(s) field)
A narrative section can be provided by the owner of the product. [ xxxing release]

\newpage
\section{Software Release Procedure} \label{sect:releaseprocedure}
Development happens in ticket branches and will be merged to master after review
Continuous integration is happening on master and provides binary packages and docker images nightly
Release process starts when all required/approved changes are merged to master and CI passes
DM CCB role: 
Review the changes included in a major release . 
Fix in Versions(s) to be used to propose issue to include in a major, minor or patch release (a Release Issue can be used also, but the native solution in jira for this use case is the Fix In Version(s) field)
In some cases DMCCB may ask the TCAM to include a specific issue into a release or
Exclude an issue from a release, and postpone it to a future major or patch release
Approve the changes to be included in patch release (Fix In Version(s))
Release technicalities remains similar to what is done (now) for the science pipeline.
SQR-16 (science pipelines) new TN on release procedures
Developer guide workflow for development activities.
Need to add in the developer guide procedure for backporting.

\newpage
\section{Incorporating third party code } \label{sec:party3}

Though the dev guide has a procedure \footnote{\url{https://developer.lsst.io/stack/packaging-third-party-eups-dependencies.html}}
for making a third part package for inclusion in the pipelines distrib we have no official policy for how we accept and deal with such packages.



There are several reasons to include third part code in our distributions:
\begin{enumerate}
\item There are packages which we use in our code e.g. starlink\_ast.  These are standard dependencies.  Currently these are put in lsst github org and probably should not be. Apart from that this is a well understood reason for third party code. \label{item:depend}
\item There are packages which we want to use when working with the deployed software. AstroPy might be such an example - we don't need it to build or run our code but its sure handy for interaction with the data.\label{item:want}
\item A collaborator has a package which depends on our code and is intended for some form of post processing. Another example here would be an algorithm we want to use such as ngmix which we do not necessarily need to support. This is what most people on LSST are thinking about under the topic of third party software. \label{item:colab}
\end{enumerate}


In all three cases the DMCCB should decide if a package is to be included in the distribution. We currently only distribute pipelines and this is the code base most likely to be affected here. One could see a need to distribute AP separately from DRP in production - the principle would remain the same a package may be included in one or more distributions. The DMCCB needs to control the list of such packages.

We currently handle dependencies such as in type \ref{item:depend} above using using EUPS and soon via conda. We could use the same mechanism for the other types of packages also,
we should require the third party contributed packages are conda installable.

\subsection{Support}\label{sec:support}
We should take care to clarify that inclusion of a package in an LSST distribution does not imply support for that package.
Packages included especially of the type \ref{item:colab} above, could be included as is not even necessarily built by our CI system nor do they need to conform to any LSST rules apart from licensing.
These should not live in the LSST github org. We currently conflate inclusion in the distrib with inclusion in the org.
We should create another github org such as lsst-community for these packages. Though if we use something like conda to install the requisite packages this is not strictly necessary.
\footnote{Even if we do support packages like starlink\_ast we should not put them under the LSST org I feel}.

\subsection{Testing} \label{sec:p3test}
In the case of type \ref{item:want} above we may want to have a certain level of testing to make sure it works when deployed.

In the case of type \ref{item:colab} above we should work with collaborators to include tests which exercises our code and interfaces in the way the code expects - this should be enough to alert us and our collaborators that the code may not work with some new release. The earlier we can catch this the better.
\subsection{Migration}\label{sec:migration}
If some collaborator code should prove very useful and is demonstrated  on say 1\% of data to provide results the community would like to have for all data then it should migrate to LSST proper and become supported. This would require work from the contributor and LSST staff to get the package to conform to an acceptable amount with LSST rules \footnote{We should have a set of acceptable rules e.g. it is essential to have unit tests, it is desirable to have a clean commit history.}.

\newpage
\appendix

\section{Status and Problems} \label{sec:statusAndProblems}

\subsection{Status of the implementation} \label{sec:status}

At the time this technote is in writing, March 2019, only the \textbf{lsst\_distrib} repository is released and distributed.


\subsubsection{Packaging} \label{sec:statusPkgs}

The following technologies are already used in DM for packaging:

\begin{itemize}
\item {\bf Eups}: for the majority of DM packages. It is used also for all third party libraries that require customization, or that are not available in the conda channels. The lsst DM binary Eups repository is hosted at \url{https://eups.lsst.codes/}. 
\item {\bf Sonatype Nexus}: for java based software packages.
\item {\bf Conda}: for packages publicly available, that can be used without any customization
\item {\bf Pip/PiPy}: for python packages publicly available.
\end{itemize}


\subsubsection{Distribution} \label{sec:statusDistrib}

The main tool used for distribution to operations is \textbf{Docker}. 

The Science Pipelines distribution to the science comunity is mainly provided using the script \textit{newinstall.sh}. 
This scripts permits to retrieve a specific Science Pipelines build or release available in the Eups repository, and setup the environment for its build and execution.

In \url{pipelines.lsst.io} instructions are provided on how to deploy a Science Pipelines using \textit{newinstall.sh} or \textbf{Docker}.


\subsubsection{SW Products Identification} \label{sec:statusIdentification}

No software products as described in the product tree in \citeds{LDM-294} have official releases so far. 

Only the Science Pipelines distribution, identified by the \textit{lsst\_distrib} Git mata-package, has regular official releases.
All explicit and implicit dependencies in \textit{lsst\_distrib} which team is \textbf{Data Management} or \textbf{DM Externals} are considered part of the distribution.

The following Git teams are relevant for the identification of the Science Pipelines distribution:

\begin{itemize}
\item the \textbf{Data Management} team identifies all DM developed software included in the distribution.
\item the \textbf{DM Externals} team identifies third-party libraries required by \textit{lsst\_distrib}.
These are software packages developed outside DM, that are not available in public conda channels, or that are updated often and therefore can't be included in the conda environment definition (\ref{sec:statusEnvs}).
See draft \citeds{DMTN-110} for current problems and possible solutions on conda environemnts.
\end{itemize}

A third team, \textbf{DM Auxiliary}, identifies auxiliary packages to be tagged when a release or build of the distribution is done, but they are not part of it, therefore not distributed with it.


\subsubsection{Environment} \label{sec:statusEnvs}

The conda environment used for building the Science Pipelines is defined in the Git repository \textit{scipipe\_conda\_env}.

\textit{TO BE CLARIFIED: how are other build environments managed, for example for java builds?}


\subsubsection{Other Tools} \label{sec:statusTools}

The build tools used to build Science Pipelines software are available in \textit{lsstsw} and \textit{lsst\_build} Git repositories.

The Jenkins scripts also are an important part of the tooling, facilitating a large number of actions, like, test a ticket branch, automate releases, or provide periodic build (weekly/daily).

The tools picture is completed by \textit{codekit}, which permits interaction with multiple Git repositories.
Just as an example, it is used to create the same tag on all the Git repositories composing a defined product automatically, instead of creating it manually.

\textit{TO BE CLARIFIED: which are the tools used with other build systems, like for example java?}


\subsection{Open Problems} \label{sec:openProblems}

In this section, a list of problems derived from the current development approach is enumerated. 

While these problems are unresolved the only release process suitable for DM is to release the entire codebase each time.

Their resolution may lead to a different development approach, and therefore the proposed release procedure may need to be reviewed accordingly.


\subsubsection{SW Product Composition} \label{sec:problemId}

A Git repository may be included in more than one software product.

This makes impossible to apply in a consistent way the release procedure to an SW product that is not the Science Pipelines.
The reason is explained in the following example.

Product A is composed of 100 Git repositories.

Product B is composed of 80 Git repositories.

70 Git repositories are shared by both products.

When release 1.0 is done on product A, all 100 repositories will be tagged, and corresponding Eups packages are created. This will include the 70 shared repositories.

One week later, product B is ready for the release 1.0, and some of the 70 shared repositories have been updated.
For them, the 1.0 tag required for product B release, will not be the same as the 1.0 tag required for the product A release.
This implies that for some repositories, release 1.0 of product A, is different from release 1.0 of product B.

\textbf{Requirement}: there should be a 1 to 1 correspondence between software products and Git repository. 
If this is not possible, each Git repository shall be included in only one software product (one SW product to many Git repositories).
Note also that, each Git repository used for building, unit testing and packaging the software products shall not be included in the SW product itself, but versioned separately and be part of the environment definition. See section \ref{sec:swprod}.


\subsubsection{Code Fragmentation} \label{sec:problemCode}

The high number of repositories causes problems in that it :

\begin{itemize}
\item increases the build time: each Git repository needs to build each time.
\item increases the release time: all Git repositories need to be built and released each time.
\item increases the failure probability: the tooling may be affected by network glitch or similar technical issues. 
This may lead to the failure of the build/release process in a non-deterministic way.
\end{itemize}

Moving 3rd party libraries to conda environment will mitigate these problem. 
However, this requires proper management of the conda environment. See draft \citeds{DMTN-110}.

\textbf{Requirement}: the number of Git repositories shall be kept low. DM-CCB shall approve each time a new repository is introduced since this has an impact on the maintainability of the system. 
All third party libraries, shall also not be part of the stack, if not requiring source code changes.


\subsubsection{Binaries Persistence} \label{sec:problemPersistence}

EUPS builds a Git repository and installs the binary packages locally. 
If a Git repository is not affected by any changes, once installed locally by EUPS, it will not be rebuilt.

Continuous integration tools instead are making available, in the remote repository at \url{https://eups.lsst.codes/}, the binary packages for macOS and Linux platforms.
However, the build tools in lsstsw are not able to resolve the binary packages from that remote repository.
Therefore, each time the Science Pipelines is built from scratch, all Git repositories are rebuilt.

A Git repository should be rebuilt, only in case its source code has changed, or one of its dependencies has changed (in case of semantic versioning, only for breaking changes that increase the major version).

\textbf{Requirement}: it shall be possible to resolve a package binaries from \url{https://eups.lsst.codes/} if available, instead of building it from the source code.

This problem is not blocking on the release process, but its resolution permits a better optimization of the builds.




\newpage
\section{Proposed Improvements}\label{sec:proimp}

Given the definitions in section \ref{sec:definitions} and the current status and problems described in appendix \ref{sec:statusAndProblems}, the following way forward is proposed.

\begin{itemize}
\item identify a software products that requires delivery soon.
A possible candidate could be the \textit{Calibration} pipeline which is now  part of the Science Pipelines, to be used soon for AuxTel commissioning.
The software product shall be defined using a \textit{metapackage} that includes only the Git packages relevant to it.
Git packages that are shared with others software products, need to be included in a different metapackage,
\item identify all the Git packages that are shared with other software products and included them in a \textit{library metapackage}.
This library metapackage is a software product itself and needs to be managed and released separately.
\item update the build system in order to be able to resolve the library from the binary repository (EUPS)
\item update the tooling to deal with tagging (code-kit) and continuous integration (Jenkins scripts).
\end{itemize}

All the Git packages included in a software product metapackage should be built and released all at the same time, as it is done now for the full Science Pipelines.

In a first stage, the above changes shall not affect the possibility to release the Science Pipelines as it is currently done.



\newpage
\section{References} \label{sec:bib}
\bibliography{lsst,lsst-dm,refs_ads,refs,books,local}


%Make sure lsst-texmf/bin/generateAcronyms.py is in your path
\section{Acronyms used in this document}\label{sec:acronyms}
\addtocounter{table}{-1}
\begin{longtable}{|l|p{0.8\textwidth}|}\hline
\textbf{Acronym} & \textbf{Description}  \\\hline

CCB & Change Control Board \\\hline
CI & Configuration Item \\\hline
DM & Data Management \\\hline
DMCCB & DM Change Control Board \\\hline
DMTN & DM Technical Note \\\hline
LDM & LSST Data Management (document handle) \\\hline
LSST & Large Synoptic Survey Telescope \\\hline
SW & Software (also denoted S/W) \\\hline
TCAM & Technical Control Account Manager \\\hline
TN & Technical Note \\\hline
\end{longtable}

\end{document}
