\documentclass[DM,authoryear,toc]{lsstdoc}
% lsstdoc documentation: https://lsst-texmf.lsst.io/lsstdoc.html

% Package imports go here.

% Local commands go here.

% To add a short-form title:
\title{DM Release Process}

% Optional subtitle
% \setDocSubtitle{A subtitle}

\author{%
Gabriele Comoretto
}

\setDocRef{DMTN-106}

\date{\today}

% Optional: name of the document's curator
% \setDocCurator{The Curator of this Document}

\setDocAbstract{%
Release procedure applicable to all Data Management SW products.
}

% Change history defined here.
% Order: oldest first.
% Fields: VERSION, DATE, DESCRIPTION, OWNER NAME.
% See LPM-51 for version number policy.
\setDocChangeRecord{%
  \addtohist{}{2019-02-04}{DM Release Process}{Gabriele Comoretto}
}

\begin{document}

% Create the title page.
% Table of contents is added automatically with the "toc" class option.

\maketitle
%switch to \maketitle if you wan the title page and toc


% ADD CONTENT HERE ... a file per section can be good for editing
\section{Introduction} \label{sec:intro}

The scope of this document is to provide a general release procedure, valid for all Software Products in the Data Management LSST subsystem. The procedure as presented here can be tailored if needed.

\subsection{Applicable Documents}

When applicable documents change a change may be required in this document.
\begin{tabbing}
AUTH-NUM\= \kill
\citeds{LDM-148} \>     DM Architecture\\
\citeds{LDM-294} \>     DM Project Management Plan   \\
%\citeds{LDM-564} \>     Data Management Releases for Verification/Integration \\
% perhaps \citell{LL:AUTH-code}\>       Software Requirements Specification for \CU,\\
\end{tabbing}

\newpage
\section{Definitions} \label{sec:definitions}

\subsection{Software Product} \label{sect:swprod}
A release is made of a SW product
A SW Product should correspond to a single repository (git package)
In the case of lsst a SW corresponds to multiple git packages but the single repository can be mimicked using a metapackage 


\siubsection{Software Release} \label{sect:swrel}
Identified by  a TAG in the SW repository
 documented with a software release note.
The TAG is created on a release branch after manual checks
A released software depends only on other released software packages.


\subsection{Software Binary Package} \label{sect:swbpkg}
Is a binary package created by building the SW with the identified release Tag 
Binary packages can be created to support multiple platforms (such as linux, osx, windows) if required.
A SW package can be created from a release candidate also, or from any other identified status  in the SW repository


\subsection{Distribution} \label{sect:distribution}
Is a collection of binaries to be deployed in a specific configuration, for operational purposes.
The same distribution, shall be used for validation and integration purposes.

\newpage
\section{Change control} \label{sect:cangecontrol}
The CCB decides the timing of a release 
Major releases shall be planned with sufficient advance notice (i.e. tied to LDM-503 Milestones)
Patch release need to be agreed by the CCB on case by case basis
CCB shall also approve the issues to be included in or back ported to a patch release.

\newpage
\section{Release note} \label{sect:relnote}
List of jira issues automatically generated from the github repo [Fix in Versions(s) field in Jira can be filled by the developer if he has the information, or can be populated a posteriori when the release is done.] 
(a Release Issue can be used also, but the native solution in Jira for this use case is the Fix in Version(s) field)
A narrative section can be provided by the owner of the product. [ xxxing release]

\newpage
\section{Software Release Procedure} \label{sect:releaseprocedure}
Development happens in ticket branches and will be merged to master after review
Continuous integration is happening on master and provides binary packages and docker images nightly
Release process starts when all required/approved changes are merged to master and CI passes
DM CCB role: 
Review the changes included in a major release . 
Fix in Versions(s) to be used to propose issue to include in a major, minor or patch release (a Release Issue can be used also, but the native solution in jira for this use case is the Fix In Version(s) field)
In some cases DMCCB may ask the TCAM to include a specific issue into a release or
Exclude an issue from a release, and postpone it to a future major or patch release
Approve the changes to be included in patch release (Fix In Version(s))
Release technicalities remains similar to what is done (now) for the science pipeline.
SQR-16 (science pipelines) new TN on release procedures
Developer guide workflow for development activities.
Need to add in the developer guide procedure for backporting.


\appendix
% Include all the relevant bib files.
% https://lsst-texmf.lsst.io/lsstdoc.html#bibliographies
\section{References} \label{sec:bib}
\bibliography{lsst,lsst-dm,refs_ads,refs,books}

%Make sure lsst-texmf/bin/generateAcronyms.py is in your path
\section{Acronyms used in this document}\label{sec:acronyms}
\addtocounter{table}{-1}
\begin{longtable}{|l|p{0.8\textwidth}|}\hline
\textbf{Acronym} & \textbf{Description}  \\\hline

CCB & Change Control Board \\\hline
CI & Configuration Item \\\hline
DM & Data Management \\\hline
DMCCB & DM Change Control Board \\\hline
DMTN & DM Technical Note \\\hline
LDM & LSST Data Management (document handle) \\\hline
LSST & Large Synoptic Survey Telescope \\\hline
SW & Software (also denoted S/W) \\\hline
TCAM & Technical Control Account Manager \\\hline
TN & Technical Note \\\hline
\end{longtable}

\end{document}